\documentclass[]{cvStyle1}

\begin{document}
    \begin{header}
        \name{Cristhian Andr\'es}{Vega Cort\'es}\\
        \major{Ingeniero en Inform\'atica}\\
        \contact
            {\faLinkedin : \href{http://www.linkedin.com/in/cristhianvega}{cristhianvega}}
            {\Mobilefone : +569 5 965 7229}
            {\Letter : cristhianv.info@gmail.com - \faGithub : \href{http://github.com/kernuac}{kernuac}}
    \end{header} 
     
    \section*{Resumen}
        \resume{
            Soy una persona que constantemente est\'a en b\'usqueda de aprender sobre 
            nuevas tecnolog\'ias y como usarlas para mejorar procesos o sistemas y 
            simplificar tareas. Tambi\'en una persona independiente que puede
            resolver problemas sin necesidad de una supervisi\'on constante.
        }  
        
    \section*{Experiencia Laboral}
        \experience{2013 -- Presente}{Ingeniero de Soporte Tecnol\'ogico - Universidad de Atacama}
        {
            \begin{description}[leftmargin=0cm]
                \item [Administrador de Sistemas] \hfill \\
                Mi primera tarea fue implementar las redes y servicios. 
                Para ello, configur\'e un {\em Firewall} con {\em Linux} e {\em Iptables}, 
                un servidor {\em Proxy}/{\em Cache} con {\em Squid}; dos Controladores 
                de Dominio con {\em Windows Server 2008}; un servidor Web y 
                un Portal Cautivo conectado a los controladores de 
                Dominio. El servidor {\em Proxy}, Servidor Web y algunos otros
                fueron virtualizados usando {\em Xen Hypervisor}. Esta 
                estructura permiti\'o controlar el Tr\'afico web y el buen 
                uso de las estaciones de trabajo, minimizando problemas 
                por virus y mantenciones correctivas de los mismos.

                \item [Desarrollador de Aplicaciones] \hfill \\
                Colabor\'e en el desarrollo de un Sistema on-line de 
                solicitudes acad\'emicas para los estudiantes de la 
                Facultad Ciencias de la Salud. Esta aplicaci\'on fue 
                escrita usando {\em PHP} en el lado del servidor, {\em MS-SQL} 
                como motor de base de datos y {\em HTML5 + CSS3 + Javascript} 
                para el FrontEnd. Con este software, se redujo el tiempo
                para la resoluci\'on de solicitudes de 15 d\'ias a 5 d\'ias.
            
            \end{description}
        }
                     
        \experience{2012 -- 2013}{Ingeniero de Soporte - Holding San Carlos}{
            \begin{description}[leftmargin=0cm]
                \item [Desarrollador de Aplicaciones] \hfill \\
                Mientras trabaj\'e en esta compa\~n\'ia, desarroll\'e algunas
                aplicaciones para el apoyo de las principales
                actividades de la misma, por ejemplo un Sistema de asignaci\'on
                y administracion de Tareas (Workflow), un sistema para
                solicitudes de Compras, etc. Cada aplicaci\'on fue escrita
                como m\'odulos para la plataforma E107 (CMS), ya
                existente, en {\em PHP}. Estas aplicaciones lograron facilitar
                la toma de decisiones y estandarizar procesos.
                
                \item [Administrador de Sistemas] \hfill \\
                Tuve que controlar el tr\'afico de red aplicando ciertas
                restricciones para acceder a internet. Para ello,
                implement\'e un {\em Firewall} con {\em Linux} e {\em Iptables}; un
                {\em Proxy/Cache} con {\em Squid} para filtrar contenido y algunos
                monitores como {\em NTOP}, para analizar el tr\'afico y
                {\em Calamaris} para analizar los Logs de {\em Squid}. Tambi\'en
                defin\'i horarios espec\'ificos para acceder a internet sin
                restricciones. Finalmente, La red se vi\'o beneficiada con
                estas implementaciones notando una mejor\'ia en la misma.
            
            \end{description}
            }
        \experience{2009 -- 2012}{Administrativo Inform\'atico - Universidad del Mar}{
                Trabajando para esta universidad, estuve a cargo de mantener los
                sistemas que estaban en funcionamiento e implementar otros nuevos
                que fueron solicitados. Estos fueron un {\em Firewall}, un controlador
                de Dominio y un Portal Cautivo. Los sistemas que ya se encontraban
                en funcionamiento eran un servidor {\em Proxy/Cache} con {\em Squid} y algunos
                servidores Web.

                En primer lugar, configur\'e un {\em Firewall} con {\em Iptables} sobre un
                equipo {\em Linux}. La siguiente tarea fue implementar un controlador
                de Dominio con {\em Windows Server 2008}. Finalmente, Trabaj\'e en
                conectar un Portal Cautivo con PF-Sense a dicho Controlador de
                Dominio. Esto fortaleci\'o la infraestructura y minimiz\'o
                problemas por virus y mantenciones correctivas a los equipos.
                
            }
            
    \section*{Formaci\'on Acad\'emica}
        \experience{INACAP}{Ingeniero en Inform\'atica}{Destacado como el Mejor Egresado de su Promoci\'on}
        
    \section*{Idiomas}
        \skill{Espa\~nol}{Nativo}
        
        \skill{Ingl\'es}{Escrito - Intermedio, Hablado - B\'asico}
    %\pagebreak
    
    %\section*{Habilidades}
    %    \skill{Sistemas Operativos}{GNU/Linux, Ms Windows}
        
    %    \skill{Leng. de Programaci\'on}{Python, C\#, PHP, C,C++, Java}
        
    %    \skill{Lenguajes de Scripting}{javascript, bash}
        
    %    \skill{Frameworks Web}{Angular.js, JQuery, CakePHP}
        
    %    \skill{Ofim\'atica}{LibreOffice, Ms-Office, Latex}
        
    %    \skill{Servicios}{Proxy Cache(SQUID), Active Directory, DNS (Bind9),
    %        Paravitualizaci\'on (Xen), Servicios Web (Apache2, Nginx),
    %        Captive portal, Bases de Datos (SQL Server, MySQL),
    %        Firewall (iptables)
    %    }
        
    %    \skill{LMS}{Moodle, Dokeos, Claroline}
        
    %    \skill{CMS}{Joomla, Drupal, e107, Wordpress, Blogger}
        
    \section*{Cursos / Diplomados}
        \skill
            {World Wide Web Consortium (W3C)}
            {Front-End Web Developer}
        \skill
            {World Wide Web Consortium (W3C)}
            {HTML5: Advanced Techniques for Designing HTML5 Apps}
        \skill
            {World Wide Web Consortium (W3C)}
            {HTML5: Coding Essentials and Best Practices}
        \skill
            {World Wide Web Consortium (W3C)}
            {HTML5: Introduction}
        \skill
            {Universidad Aut\'onoma de Madrid}
            {Jugando con Android - Aprende a programar tu primera App}
        \skill
            {CodeAchademy}
            {Angular.js}
        \skill
            {CodeSchool}
            {Shaping up with Angular.js}
        \skill
            {Uiversidad Cat\'olica de Chile}
            {Evaluaci\'on de Decisiones Estrat\'egicas}
\end{document}
