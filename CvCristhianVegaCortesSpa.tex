\documentclass[]{cvStyle1}

\begin{document}
    \begin{header}
        \name{Cristhian Andr\'es}{Vega Cort\'es}\\
        \major{Ingeniero en Inform\'atica}\\
        \contact
			{\PointingHand 16.326.682-2}
			{\Mobilefone +569 8 2671793}
			{\Letter cristhianv.info@gmail.com}
    \end{header} 
     
    \section*{Resumen}
        \resume{
            Profesional autodidacta, con capacidad para trabajar en equipo y
            establecer buenas relaciones con sus pares. Destaca su
            puntualidad, responsabilidad y compromiso para cumplir metas.
        }  
        
    \section*{Experiencia Laboral}
        \experience{2013 -- Presente}{Ingeniero de Soporte Tecnol\'ogico - Universidad de Atacama}
        {
            \begin{description}[leftmargin=0cm]
                \item [Administrador de Sistemas] \hfill \\
                    Como Administrador de sistemas en la 
                    Facultad Ciencias de la Salud, de la Universidad de
                    Atacama, administro una red de m\'as de 250 
                    computadores con m\'as de 1000 cuentas de usuario.
                    Para ello, implement\'e un controlador de
                    dominio sobre Windows Server 2008.
                    Con esto se redujo la tasa de mantenciones
                    correctivas a un 2\% y las infecciones por virtus o 
                    software malicioso a 0. Adem\'as, la centralizaci\'on
                    de perfiles de usuario disminuy\'o el tiempo de puesta
                    en marcha de una estaci\'on de trabajo.

                \item [Desarrollador de Aplicaciones] \hfill \\
                    Como desarrollador de aplicaciones en esta misma
                    Facultad, desarroll\'e un sistema para gestionar
                    solicitudes acad\'emicas que, anteriormente, se
                    realizaban en papel.
                    Este software fue desarrollado utilizando
                    PHP en el lado del Servidor, html5 para estructurar
                    la informaci\'on, css + bootstrap para el aspecto
                    visual y javascript para manejar los eventos de la
                    interfaz web.
                    Con \'Esto se logr\'o disminuir el tiempo de 
                    tramitaci\'on de 15 d\'ias a 5 d\'ias,
                    descongestionando bastante el proceso.
 
                \item [Soporte Tecnol\'ogico] \hfill \\
                    Situacion:
                    Tarea: realic\'e mantenciones preventivas y
                    correctivas a las estaciones de trabajo de 
                    las empresas relacionadas con el Holding.
                    Accion: Para ello,
                    Resultados:
            \end{description}
        }
                     
        \experience{2012 -- 2013}{Ingeniero de Soporte - Holding San Carlos}{
			\begin{description}[leftmargin=0cm]
				\item [Desarrollador de Aplicaciones] \hfill \\
                Como desarrollador de aplicaciones en Holding San Carlos
                sistematic\'e el flujo de actividades dentro de la
                empresa, el proceso de solicitudes de compra, 
                Para ello desarroll\'e m\'odulos para la 
                plataforma ya existente, e107, escritos en php.
                Con esto se agiliz\'o la toma de decisiones en
                las \'Areas que fueron intervenidas.
                
                \item [Administrador de Sistemas] \hfill \\
                En esta \'area, estuve a cargo de administrar las
                redes y servicios de las instalaciones principales
                del Holding. 
                Implement\'e un peque\~no firewall con iptables, un
                servidor proxy/cache con squid para filtrar contenido
                y definir horarios de uso del recurso internet
                mediante acls y sistemas de monitoreo de red.
                Gracias a ello, se optimiz\'o el uso de internet.

                \item [Soporte Tecnol\'ogico] \hfill \\
            
            \end{description}
            }
        \experience{2009 -- 2012}{Administrativo Inform\'atico - Universidad del Mar}{
                Administrador de una red de 250 computadores, controladas
                bajo un Controlador de Dominio Windows 2008 Server. 
                Gesti\'on de mas de 1200 perfiles de usuarios mediante
                pol\'ticas de grupo con permisos diferenciados seg\'un
                necesidades de la instituci\'on. Tales pol\'iticas
                redujeron la tasa de mantenciones correctivas a los
                equipos a un 2\% en 3 a\~nos de implementaci\'on.
            }
            
    \section*{Formaci\'on Acad\'emica}
        \experience{INACAP}{Ingeniero en Inform\'atica}{Destacado como el Mejor Egresado de su Promoci\'on}
        
    \section*{Idiomas}
        \skill{Espa\~nol}{Nativo}
        
        \skill{Ingl\'es}{Escrito - Intermedio, Hablado - B\'asico}
    %\pagebreak
    
    %\section*{Habilidades}
    %    \skill{Sistemas Operativos}{GNU/Linux, Ms Windows}
        
    %    \skill{Leng. de Programaci\'on}{Python, C\#, PHP, C,C++, Java}
        
    %    \skill{Lenguajes de Scripting}{javascript, bash}
        
    %    \skill{Frameworks Web}{Angular.js, JQuery, CakePHP}
        
    %    \skill{Ofim\'atica}{LibreOffice, Ms-Office, Latex}
        
    %    \skill{Servicios}{Proxy Cache(SQUID), Active Directory, DNS (Bind9),
    %        Paravitualizaci\'on (Xen), Servicios Web (Apache2, Nginx),
    %        Captive portal, Bases de Datos (SQL Server, MySQL),
    %        Firewall (iptables)
    %    }
        
    %    \skill{LMS}{Moodle, Dokeos, Claroline}
        
    %    \skill{CMS}{Joomla, Drupal, e107, Wordpress, Blogger}
        
    \section*{Cursos / Diplomados}
        \skill
			{Uiversidad Cat\'olica de Chile}
			{Evaluaci\'on de Decisiones Estrat\'egicas}
              
        \skill
			{CodeSchool}
			{Shaping up with Angular.js}
        
        \skill
			{}
			{Staying Sharp with Angular.js (En curso)}
        
        \skill
			{CodeAchademy}
			{Angular.js}
        
        \skill
			{Universidad Aut\'onoma de Madrid}
			{Jugando con Android - Aprende a programar tu primera App}
        
        \skill
			{World Wide Web Consortium (W3C)}
			{HTML5 Introduction}
        
        \skill
			{World Wide Web Consortium (W3C)}
            {HTML5 Coding Essentials and Best Practices}
            
        \skill
			{World Wide Web Consortium (W3C)}
			{HTML5 }
            
\end{document}
