\documentclass[]{cvStyle1}

\begin{document}
    \begin{header}
        \name{Cristhian Andr\'es}{Vega Cort\'es}\\
        \major{Informatics Engineer}\\
        \contact{\PointingHand 16.326.682-2}{\Mobilefone +569 8 2671793}{\Letter cristhianv.info@gmail.com -- kernuac}
    \end{header} 
     
    \section*{Resumen}
        \resume{
	    I'm a self-taught person who can... 
            %Profesional autodidacta, con capacidad para trabajar en equipo y
            %establecer buenas relaciones con sus pares. Destaca su
            %puntualidad, responsabilidad y compromiso para cumplir metas.
        }  
        
    \section*{Experiencia Laboral}
        \experience{2013 -- Present}{Technology Support Engineer - Universidad de Atacama - Chile}{
             \begin{description}[leftmargin=0cm]
                \item [- Developer] \hfill \\
                As a developer at Universidad de Atacama, Chile, I collaborated in the developement
                of an On-line Achademic Requests System for our students, This application was written using 
                {\em PHP} as server side language, {\em MS-SQL Server} for storing data and {\em HTML5 + CSS3 +
                Javascript} for the frontend. With this new software, the time for resolving students 
                request decreased from 15 days to 5 days.
                
                \item [- Systems Administrator] \hfill \\
                When the new facultad was created, the first task was to implement networks and
                services. I had to implement a firewall with linux and iptables, 
                a proxy/cache server with squid, two Domain controllers on {\em Windows Server 2008}, 
                a Web server and a captive portal connected with active directory on the Domain
                Controllers. The Proxy Server, Web Server, and some others were virtualized using 
                {\em Xen hypervisor}. This structure allowed to control network traffic and the good use
                of workstations, minimizing issues for viruses and corrective maintenances.
             \end{description}
        }
        \experience{2012 -- 2013}{Technical Support Engineer - Holding San Carlos - Chile}{
            \begin{description}[leftmargin=0cm]
                \item [- Developer] \hfill \\
                While I was working for this company, I created some applications for supporting the
                main activities there, for example: a Workflow System, a Purchase Requests System,
                etc. Each application was written as modules for {\em e107-CMS} in {\em PHP}.
                These applications helped to take decisions faster and standardize process.
                
                \item [- Systems Administrator] \hfill \\
                On the other Hand, I had to control the network traffic and apply some restrictions for
                accessing to Internet. So, I implemented a little firewall on a {\em Linux} machine
                with {\em iptables}, a Proxy/Cache with {\em Squid}, for filtering content,
                and some monitors like {\em NTOP} for analizing traffic and {\em Calamaris} for 
                analizing Squid's logs. Also, I defined a range of hours when workers can access to
                internet with out restrictions. Finally, the network traffic was optimized after
                that.
            \end{description}
        }
        \experience{2009 -- 2012}{Technical Support Engineer - Universidad del Mar - Chile}{
            \begin{description}[leftmargin=0cm]
                \item [- Systems Administrator] \hfill \\
                Working for this university, I had to maintain current systems and implementing a 
                Firewall, a Domain controller and a Captive Portal. Current systems was a Proxy
                Cache with {\em SQUID} and, a web server. First, I configured a firewall with iptables on a Linux
                machine. The next task was to implemet a Domain controller on {\em Windows Server 2008}.
                Finally, I worked in a Captive Portal using {\em PFSense} and connected to the Domain
                Controller. This made stronger our infrastructure and minimized issues for viruses
                and corrective maintenances of workstations.
            \end{description}
        }
            
    \section*{Formaci\'on Acad\'emica}
        \experience{INACAP}{Informatics Engineer}{
            My final project was a 2D video game prototype written in {\em python} and using 
            {\em pygame} library. This little adventure game tried to help students up to 8 years 
            to practice basic arithmetic calculations such as adittion, substraction, multiplication
            and division.
        }
        
    \section*{Idiomas}
        \skill{Spanish}{Native}
        \skill{English}{Writting: Intermediate, Speaking: Beginner }
    %\pagebreak
    %\section*{Habilidades}
    %    \skill{Sistemas Operativos}{GNU/Linux, Ms Windows}
    %    \skill{Leng. de Programaci\'on}{Python, C\#, PHP, C,C++, Java}
    %    \skill{Lenguajes de Scripting}{javascript, bash}
    %    \skill{Frameworks Web}{Angular.js, JQuery, CakePHP}
    %    \skill{Ofim\'atica}{LibreOffice, Ms-Office, Latex}
    %    \skill{Servicios}{Proxy Cache(SQUID), Active Directory, DNS (Bind9),
    %        Paravitualización (Xen), Servicios Web (Apache2, Nginx),
    %        Captive portal, Bases de Datos (SQL Server, MySQL),
    %        Firewall (iptables)
    %    }        
    %    \skill{LMS}{Moodle, Dokeos, Claroline}        
    %    \skill{CMS}{Joomla, Drupal, e107, Wordpress, Blogger}      
    \section*{Courses / Certifications}
        \skill{Universidad Cat\'olica de Chile}{Evaluaci\'on de Decisiones
              Estrat\'egicas}
              
        \skill{CodeSchool}
	    {
            \vspace{-7mm}
            \begin{itemize}[leftmargin=0cm,label={-}]
            \item Shaping up with Angular.js
            \item Staying Sharp with Angular.js (Coursing)
            \end{itemize}
        }
        
        %\skill{}{Staying Sharp with Angular.js (En curso)}
        
        \skill{CodeAchademy}
	    {Angular.js}
        
        \skill{Universidad Aut\'onoma de Madrid}
	    {Jugando con Android - Aprende a programar tu primera App}
        
        \skill{World Wide Web Consortium}
        {
            \vspace{-7mm}
            \begin{itemize}[leftmargin=0cm,label={-}]
                \item HTML5 Introduction.
                \item HTML5 Part 1: Coding Essentials and Best Practices.
                \item HTML5 Part 2: Advanced Techinques for Designing HTML5 Apps.
            \end{itemize}
        }
            
\end{document}
