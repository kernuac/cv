\documentclass[]{cvStyle1}

\begin{document}
    \begin{header}
        \name{Cristhian Andr\'es}{Vega Cort\'es}\\
        \major{Ingeniero en Inform\'atica}\\
        \contact{\PointingHand 16.326.682-2}{\Mobilefone +569 8 2671793}{\Letter cristhianv.info@gmail.com}
    \end{header} 
     
    \section*{Resumen}
        \resume{
            Profesional autodidacta, con capacidad para trabajar en equipo y
            establecer buenas relaciones con sus pares. Destaca su
            puntualidad, responsabilidad y compromiso para cumplir metas.
        }  
        
    \section*{Experiencia Laboral}
        \experience{2013 -- Presente}{Ingeniero de Soporte Tecnol\'ogico - Universidad de Atacama}{
                Administrador de una red de m\'as de 250 computadores,
                controlados bajo Windows 2008 Server. Gesti\'on de
                m\'as de 1000 perfiles de usuario mediante pol\'iticas
                de grupo con permisos diferenciados seg\'un criterios
                definidos. Éstas póliticas redujeron la tasa
                mantenciones correctivas a un 2\% y las infecciones por
                virus y software malicioso al mínimo en 2 a\~nos de
                implementaci\'on. 
                La centralizaci\'on de perfiles de usuario,
                para la plana administrativa, permiti\'o la disminuci\'on en
                tiempo de puesta en marcha de las estaciones de trabajo,
                por cambio de equipo de diversa \'indole, y la
                automatizaci\'on de los sistemas de Respaldo de dichos
                perfiles.
            }
                     
        \experience{2012 -- 2013}{Ingeniero de Soporte - Holding San Carlos}{
                Administraci\'on General de Redes y Servicios
                Desarrollo de aplicaciones PHP + MySQL para el
                apoyo a la gesti\'on y toma de decisiones por
                parte de la gerencia.\\
                Soporte y Mantenci\'on a estaciones de trabajo 
                MS-Windows.
            }
        \experience{2009 -- 2012}{Administrativo Inform\'atico - Universidad del Mar}{
                Administrador de una red de 250 computadores, controladas
                bajo un Controlador de Dominio Windows 2008 Server. 
                Gesti\'on de mas de 1200 perfiles de usuarios mediante
                pol\'ticas de grupo con permisos diferenciados seg\'un
                necesidades de la instituci\'on. Tales pol\'iticas
                redujeron la tasa de mantenciones correctivas a los
                equipos a un 2\% en 3 a\~nos de implementaci\'on.
            }
            
    \section*{Formaci\'on Acad\'emica}
        \experience{INACAP}{Ingeniero en Inform\'atica}{Destacado como el Mejor Egresado de su Promoci\'on}
        
    \section*{Idiomas}
        \skill{Espa\~nol}{Nativo}
        
        \skill{Ingl\'es}{Escrito - Intermedio, Hablado - B\'asico}
    \pagebreak
    
    \section*{Habilidades}
        \skill{Sistemas Operativos}{GNU/Linux, Ms Windows}
        
        \skill{Leng. de Programaci\'on}{Python, C\#, PHP, C,C++, Java}
        
        \skill{Lenguajes de Scripting}{javascript, bash}
        
        \skill{Frameworks Web}{Angular.js, JQuery, CakePHP}
        
        \skill{Ofim\'atica}{LibreOffice, Ms-Office, Latex}
        
        \skill{Servicios}{Proxy Cache(SQUID), Active Directory, DNS (Bind9),
            Paravitualización (Xen), Servicios Web (Apache2, Nginx),
            Captive portal, Bases de Datos (SQL Server, MySQL),
            Firewall (iptables)
        }
        
        \skill{LMS}{Moodle, Dokeos, Claroline}
        
        \skill{CMS}{Joomla, Drupal, e107, Wordpress, Blogger}
        
    \section*{Cursos / Diplomados}
        \skill{Pont. Universidad Cat\'olica de Chile}{Evaluaci\'on de Decisiones
              Estrat\'egicas}
              
        \skill{CodeSchool}{Shaping up with Angular.js}
        
        \skill{}{Staying Sharp with Angular.js (En curso)}
        
        \skill{CodeAchademy}{Angular.js}
        
        \skill{U. Aut\'onoma de Madrid}{Jugando con Android - Aprende a programar tu primera App}
        
        \skill{World Wide Web Consortium (W3C)}{HTML5 Introduction}
        
        \skill{World Wide Web Consortium (W3C)}
            {HTML5 Coding Essentials and Best Practices}
            
\end{document}
